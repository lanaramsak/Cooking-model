
In the past years the industry put an emphasise on being the fastest and most efficient. Takeaway is getting more and more popular. Next day delivery is not as rare as it used to be and products are rarely hand made. 
Optimizing the process plays a big role in that. A simple example can be cooking, where the order of tasks significantly impacts the efficiency. When following a recipe, you have to think in advance what you will need, and how much time it will take. If you don't turn on the oven before starting, you might waste time waiting for it to heat up later. In our project we will be trying to find an optimal ordering of tasks, to make the cooking process more efficient. This way recipes can be done quicker and the work can be better distributed. \\

For our starter recipe simulation we chose something basic. The recipe that came to mind was the dish goulash. Here is the following recipe written in steps:
\begin{itemize}
    \item cutting the onions
    \item frying the onions
    \item cutting the meat 
    \item cooking the meat 
    \item peeling potatoes
    \item cooking potatoes in the previously boiled water
    \item cooking everything together
\end{itemize}

As you can see it includes a lot of different ingredients and cutting tasks, which are fairly simple to include in our simulation but help set the start model. 
Furthermore, it is a recipe where the order of the tasks is important, which is something you have to be aware of when cooking. 

Throughout the project we also implemented two other recipes, potato salad and cake. One was highly based on a cooking book and was 
used to test wheather our order is equal to the one in the book. Lastly, we added the recipe for the cake, to better observe the 
influence of different numbers of cooks.
