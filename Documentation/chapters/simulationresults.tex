For the simulation results we looked into a few different problematic topics.

\subsection{Effect of different ordering}
We wanted to see how big of an impact has the order of the tasks. That is why we looked for the best order of the recipe for goulash, and the worst one. 
Meaning the one that is done the quickest and the one that takes the longest.
The following results are done with two cooks and time randomness.

\begin{verbnobox}[\fontsize{10pt}{10pt}\selectfont]
BEST ORDER: 168, 
[cutting meat, peeling potatoes, heating water, cutting onions, cooking meat, 
cooking potatoes, frying onions, cooking everything]
\end{verbnobox}

\begin{verbnobox}[\fontsize{10pt}{10pt}\selectfont]
WORST ORDER: 253,
[cutting onions, cutting meat, heating water, cooking meat,
 frying onions, peeling potatoes, cooking potatoes, cooking everything]
\end{verbnobox}

We can notice that the difference in duration is 85 units of time, which is not small. 
This happens due to probalistic duration of time. Even without that, the difference would still be 60 units. The reason for that is the order of the first few tasks, 
which have to be done before cooking potatoes and cooking everything. If they are not done in the best order, the simulation has to wait for a task to be finished before doing anything new. 

\begin{figure}[H]
    \centerline{\includegraphics[scale=.4]{/Users/lanar/Documents/Cooking-model/Documentation/images/recipes_order.png}}
    \caption{The best and worst order of tasks.}
    \label{fig3}
\end{figure}

After seeing the difference an order of tasks can have, we were interested to see what are all possible durations of the recipe and which are the most common. 
In the histogram \ref{fig4} below we can make out that most of the time the recipe is completed in between 176 to 186 units of time.

\begin{figure}[H]
    \centerline{\includegraphics[scale=.5]{/Users/lanar/Documents/Cooking-model/Documentation/images/graph_duration.png}}
    \caption{All possible durations of the recipe with two cooks, devided into groups by time interval 10 units.}
    \label{fig4}
\end{figure}

We also did the same calculations for just one cook, but got a somewhat different distrubution, with most of the orders gathered in the middle, and there is not one interval that would significantly stand out. 
Most orders of tasks were completed in 240 to 260 units of time. This corelates with the optimal time for one cook, which is 251 units of time.

\begin{figure}[H]
    \centerline{\includegraphics[scale=.5]{/Users/lanar/Documents/Cooking-model/Documentation/images/graph_duration1.png}}
    \caption{All possible durations of the recipe with one cook, devided into groups by time interval 10 units.}
    \label{fig5}
\end{figure}

\newpage
\subsection{Recipe comparison to cook book}
For our second dish we decided to do potato salad. We found a recipe in a cook book \cite[page 42]{book} and adjusted it to our program. We wanted to test if the calculated optimal order for one cook matches the order in the cook book.
Here we can see the two orders:

\begin{verbnobox}[\fontsize{10pt}{10pt}\selectfont]
    COOK BOOK:                                  OPTIMAL ORDER:
    - cooking potatoes                          - cooking potatoes
    - cooking eggs                              - cooking eggs
    - cutting pickles                           - cutting onions
    - cooling eggs                              - make dressing
    - cutting onions                            - cutting pickles
    - peeling eggs                              - cooling eggs
    - cutting eggs                              - peeling eggs
    - peeling potatoes                          - peeling potatoes
    - cuttting potatoes                         - cutting potatoes
    - make dressing                             - cutting eggs
    - mixing enerything                         - mixing enerything
\end{verbnobox}

We can spot that the two orders do not differ much. Some tasks are done in a different order, but those are mainly interchangable. One different thing is when the dressing is done. 
This is because in real life you would want it to be freshly done before completing the salad. Unfortunately, our program does not take this into account. 

\subsection{Different number of resources}
Lastly, we focused on the number of resources to see if it really has that big of an effect on duration. 
We ran the program 200 times for each number of cooks and took the average of the optimal time. We gave the other resources a high enough quantity so that it will not affect the duration.
These are the graphs we got for the recipe goulash \ref{fig6} and for potato salad \ref{fig7}.

\begin{figure}[H]
    \centering
    \begin{minipage}{.5\textwidth}
      \centering
      \includegraphics[width=.9\linewidth]{/Users/lanar/Documents/Cooking-model/Documentation/images/optimal_durationG.png}
      \caption{Duration of recipe goulash with \\ different number of cooks,\\ calculated average over 200 iterations.}
      \label{fig6}
    \end{minipage}%
    \begin{minipage}{.5\textwidth}
      \centering
      \includegraphics[width=.9\linewidth]{/Users/lanar/Documents/Cooking-model/Documentation/images/optimal_durationS.png}
      \caption{Duration of recipe potato salad with different number of cooks, calculated average over 200 iterations.}
      \label{fig7}
    \end{minipage}
\end{figure}

As you can see, there is only a difference between using one or two cooks. After that you can use as many as you want, but the duration will not change. 
This happens because the tasks in our recipes already have an order in which they have to be done. Even if you have another cook, he still has to wait for the other task to finish. 
To challenge our program we decided to make another recipe, where we tried to construct it so, that a different number of cooks will have a bigger effect.

\begin{figure}[H]
    \centerline{\includegraphics[scale=.5]{/Users/lanar/Documents/Cooking-model/Documentation/images/graph_durationC.png}}
    \caption{Duration of recipe cake with different number of cooks, calculated average over 50 iterations.}
    \label{fig8}
\end{figure}

These \ref{fig8} are the results we got with implementing the recipe of baking a cake. We wanted to have more independant tasks and less strict prior ordering. 
In the recipe you have to do a lot of tasks independantly and just a few tasks that combine all of them. This way you can spot a difference 
between one, two and three cooks. However, when it comes to four, the duration still remains the same. Furthermore the difference was very small, only 6 time units. This brings us to the conclusion that with every recipe 
there will be a point where the number of cooks does not matter anymore. 

%ADD GRAPH FOR POTATO SALAD, ONLY ONE OF EVERY RESOURCE

Besides observing the effect of the number of cooks, we did the same for other resources. Following you can see the duration for making goulash when having:
\begin{itemize}
    \item double of everything: 181
    \item only one pan: 191
    \item only one knife: 181
\end{itemize}

This happens because, as you can see on figure \ref{fig9}, in the optimal recipe two knives are never used at the same time, while the pan is used for cooking the meat and frying the onions which is done at the same time. 

\begin{figure}[H]
    \centerline{\includegraphics[scale=.4]{/Users/lanar/Documents/Cooking-model/Documentation/images/resources.png}}
    \caption{Use of pan and knife in the optimal order of the recipe goulash.}
    \label{fig9}
\end{figure}
